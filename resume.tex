\documentclass[11pt,a4paper]{moderncv}
\moderncvtheme[green]{casual}
\usepackage[utf8]{inputenc}
\usepackage[pdftex]{hyperref}
\usepackage[scale=0.8]{geometry}
\usepackage{color}

\firstname{Ivan}
\familyname{Chernetsky}
\address{48, Plehanova str. Apt 5}{Minsk, Belarus 220095}
\mobile{+375~(29)~1115496}
\email{ivan.chernetsky@gmail.com}
\quote{Software Engineer Looking for Challenging Position}
\photo[64pt]{picture}

\begin{document}

\maketitle

\section{Education}
\cventry{2005--2010}{Engineer's degree}{BSUIR}{Minsk}{}{Graduated from \emph{Belarusian State University of Informatics and Radioelectronics} in \emph{2010}. The professional qualification received is \emph{mathematician--software engineer}.}

\section{Diploma project}
\cvline{title}{\emph{Methods of recognition, detection and tracking of moving objects and its application to people tracking}}
\cvline{description}{The goal of the diploma project was to develop an application for human detection and pose estimation using \emph{pictorial structures framework}. I based my work on the paper ``Pictorial structures revisited: people detection and articulated pose estimation'' by M. Andriluka, S. Roth, and B. Schiele presented at CVPR'09.}
\cvline{tools}{\emph{Linux, C/C++, Python, Google Protocol Buffers, CMake, OpenCV}}

\section{Teaching}
\cvline{09.2010--\\present}{My friend and I conduct a course on programming languages and technologies. Some topics are Introduction to Unix and its Environment, Basic Data Structures and Algorithms, Parallel Programming, Network Programming, Python, Functional Programming using OCaml, Advanced Data Structures, Common Lisp. The lecture notes can be found at \url{http://goo.gl/MeLE3}}

\section{Experience}
\cventry{08.2010--01.2012}{Senior Software Engineer}{EPAM Systems}{Minsk}{}{The main goal of the project I was working on, was to simplify — in terms of running time and a number of queries — an existing tool that pulled Merger \& Acquisition deals from an editorial database, made some calculations, and pushed a new pile of data into distribution databases. I was coding new functions for financial calculations in C and was optimizing queries on huge tables.}
\cvline{tools}{\emph{Solaris, C, Oracle, GNU make, shell}}

\cventry{08.2010--12.2010}{Mobile developer of InviteBox.com}{InviteBox}{Minsk}{}{InviteBox is a widget for swift referral promotions over social networks, email, and mobile phones. InviteBox allows your users to share your product or service with their existing social graph. By signing in with a Gmail, Facebook, Twitter or other social networking account, users can “tell a friend” to earn instant rewards and, in the process, send you high-quality traffic. It was a part-time job. I was responsible for developing widgets for iOS and Android platforms, that could be embedded in an application with a couple of lines of code.}
\cvline{tools}{\emph{Ojbective C, iOS, Java, Android, Python, Django}}

\cventry{05.2009--05.2010}{Key developer of Dooster.net}{Y-NODE Software}{Minsk}{}{Web-based project management tool. The distinctive feature of this tool is ability to control and use the most of its functionality via email. Implementation of new features, bug-fixing, support of legacy code.}
\cvline{tools}{\emph{Linux, Python, Django, jQuery, Amazon Web Services, PostgreSQL, memchached, git}}

\cventry{12.2008--05.2009}{Developer of Ahaoho.de}{Y-NODE Software}{Minsk}{}{A real postcard site. Ahaoho.de allows users to print and send real postcards. Users have an ability to choose between ready postcards and creating their own unique design by the instrumentality of the powerful online graphical editor. Optimization of statistics gathering, SEO, bug-fixing, support of legacy code.}
\cvline{tools}{\emph{Linux, Python, Django, jQuery, PostgreSQL, memcached, SVN}}

\cventry{09.2008--11.2008}{Computer vision engineer}{Y-NODE Software}{Minsk}{}{Development of software that detects simple geometric figures, training Gentle AdaBoost classifier, investigation of people detection and tracking approaches, cross-compiling, building rootfs using Buildroot. Application that performs affine transformations on selected image parts.}
\cvline{tools}{\emph{Linux, C/C++, Python, PyGTK, Buildroot, GNU make, shell, git}}

\cventry{05.2008--08.2008}{Developer of ToSeeMo.org}{Y-NODE Software}{Minsk}{}{Social networking and information marketplace site. ToSeeMo.org is an innovative service that provides means for conducting surveys while maintaining ultimate anonymity of respondents and trading the survey results between community members. Integration of authorization system with OpenID. Implementation of full-text search using Sphinx engine, development of tagging and rating applications, participation in development of Survey system.}
\cvline{tools}{\emph{Linux, Django, Python, Sphinx search, jQuery, git}}

\cventry{03.2008--05.2008}{Developer of Y-NODE.com}{Y-NODE Software}{Minsk}{}{Official company site. Full development of www.y-node.com, network programming, development of Jabber/XMPP client.}
\cvline{tools}{\emph{Linux, Django, Python, jQuery, PyXMPP, Subversion}}

\section{On the Web}
\cvlistitem{\url{http://github.com/ichernetsky}}
\cvlistitem{\url{http://about.me/ichernetsky}}

\section{Languages}
\cvline{\textbf{English}}{Upper-Intermediate}
\cvline{\textbf{Russian}}{Native}

\section{Tools used at work}
\cvcomputer{\textbf{Languages}}{C, Python, JavaScript}{\textbf{Operating Systems}}{Linux, Solaris}
\cvcomputer{\textbf{Databases}}{PostgreSQL, SQLite}{\textbf{Web Development}}{Django, jQuery, AWS}
\cvcomputer{\textbf{GUI}}{GTK+}{\textbf{Web Servers}}{nginx}
\cvcomputer{\textbf{Computer Vision}}{OpenCV}{\textbf{Mobile Development}}{iOS, Android}
\cvcomputer{\textbf{Revision Control}}{git, Subversion, CVS}{\textbf{Build Systems}}{GNU make}

\section{Tools used in personal/educational projects}
\cvcomputer{\textbf{Languages}}{Java, OCaml, Common Lisp, C++, Clojure, Erlang, Scheme, Haskell, assembly language}{\textbf{Web Development}}{AppEngine}
\cvcomputer{\textbf{Databases}}{MySQL, NoSQL}{\textbf{Build Systems}}{CMake, Buildroot, Autotools}
\cvcomputer{\textbf{GUI}}{Qt, Cairo, Clutter}{\textbf{Revision Control}}{Mercurial}

\section{Interests}
\cvlistdoubleitem{Functional programming}{Machine learning}
\cvlistdoubleitem{Image processing}{Web development}
\cvlistdoubleitem{Embedded systems}{Mobile development}
\cvlistdoubleitem{Computer vision}{Distributed systems}

\end{document}
